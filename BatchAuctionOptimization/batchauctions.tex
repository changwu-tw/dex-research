\documentclass[11pt,parskip=full]{scrartcl}%article}
%\setlength{\parskip}{1em}

%%%%%%%%%%%%%%%%%%%%%%
%% PACKAGE INCLUDES %%
%%%%%%%%%%%%%%%%%%%%%%

\usepackage{array}           % For defining \newcolumntype.
\usepackage{amsmath}
\usepackage{amssymb}
\usepackage{amsthm}          % Provides 'proof' environment.
\usepackage[english]{babel}  % For defining 'theorem/corollary/lemma' environments.
\usepackage{bm}              % Provides bold \pi
\usepackage{booktabs}
\usepackage{hyperref}
\usepackage{gensymb}         % Enables \degree command for °C.
\usepackage{todonotes}
\usepackage{subfig}
\usepackage{enumitem}


%%%%%%%%%%%%%%%%%
%% STYLE SETUP %%
%%%%%%%%%%%%%%%%%

% Define some custom colors.
\definecolor{mylinkcolor}{RGB}{000, 114, 166}
\definecolor{mycitecolor}{RGB}{255, 154, 071}
\definecolor{myurlcolor}{RGB}{000, 114, 166}

% Set itemize format.
\setitemize{noitemsep,topsep=-5pt,parsep=5pt,partopsep=0pt}

% Define column types that allow fixed width params.
\newcolumntype{L}[1]{>{\raggedright\let\newline\\\arraybackslash\hspace{0pt}}m{#1}}
\newcolumntype{C}[1]{>{\centering\let\newline\\\arraybackslash\hspace{0pt}}m{#1}}
\newcolumntype{R}[1]{>{\raggedleft\let\newline\\\arraybackslash\hspace{0pt}}m{#1}}

% Color setup for hyperlinks/references/citations/urls.
\hypersetup{
    colorlinks,
    linkcolor={mylinkcolor},
    citecolor={mycitecolor},
    urlcolor={myurlcolor}
}

% Specify hyphenation of words on line break.
\hyphenation{Figure Table Chapter Section}



%%%%%%%%%%%%
%% MACROS %%
%%%%%%%%%%%%

% Set default font family to sans-serif.
\renewcommand*{\familydefault}{\sfdefault}

\newcommand*{\ie}{i.e., }
\newcommand*{\eg}{e.g., }
\newcommand*{\wrt}{w.r.t. }

\newcommand*{\tokens}{\mathcal{T}}          % Set of tokens.
\newcommand*{\orders}{\mathcal{O}}          % Set of orders.
\newcommand*{\itokens}{\mathcal{I}^t}       % Set of token indices.
\newcommand*{\itokenpairs}{\mathcal{I}^p}   % Set of token index pairs.
\newcommand*{\iorders}{\mathcal{I}^o}       % Set of order indices.
\newcommand*{\ibuyorders}{\mathcal{I}^b}    % Set of buy order indices.
\newcommand*{\isellorders}{\mathcal{I}^s}   % Set of sell order indices.

% Macros for references etc.
\newcommand*{\figref}[1]{\hyperref[{#1}]{Figure~\ref*{#1}}}
\newcommand*{\tabref}[1]{\hyperref[{#1}]{Table~\ref*{#1}}}
\newcommand*{\secref}[1]{\hyperref[{#1}]{Section~\ref*{#1}}}
\newcommand*{\subsecref}[1]{\hyperref[{#1}]{Section~\ref*{#1}}}
\newcommand*{\thmref}[1]{\hyperref[{#1}]{Theorem~\ref*{#1}}}
\newcommand*{\crlref}[1]{\hyperref[{#1}]{Corollary~\ref*{#1}}}
\newcommand*{\lemref}[1]{\hyperref[{#1}]{Lemma~\ref*{#1}}}

% Macros for theorems|corollaries|lemmas.
\newtheorem{theorem}{Theorem}[section]
\newtheorem{corollary}{Corollary}[theorem]
\newtheorem{lemma}[theorem]{Lemma}



%%%%%%%%%%%%%%
%% DOCUMENT %%
%%%%%%%%%%%%%%

\title{
  Multi-token Batch Auctions with Uniform Clearing Prices\\
  - \\
  \Large Features and Models}
\author{Tom Walther \\ tom@gnosis.pm}

\date{\today}



\begin{document}

\maketitle


\begin{abstract}
... abstract ...
\footnote{... footnote ...}.
\end{abstract}

\todo[inline,caption={}]{
  \begin{itemize}
    \item write abstract
    \item change t for tokens to $ \tau $ everywhere
    \item change o for orders to $ \omega $ everywhere
    \item add additional inequalities for MIP1
    \item write down MIP3
    \item write something about min/max fluctuation
    \item computational results for MIP\{2|3\}
    \item choice of reference token example
    \item pictures of order-token-graph
  \end{itemize}
}

\newpage
\tableofcontents

\newpage
\section{Introduction}
\label{sec:introduction}

In continuous-time token exchange mechanisms, orders are typically collected in order books of two
tokens that are traded against each other.
A trade happens whenever a buy order of one token is matched by a sell order of the other, \ie if
there exists an exchange rate that satisfies the limit prices stated in the respective orders.
In a setting of multiple tradeable tokens, one separate order book is required for every token pair
combination.
This may significantly limit liquidity for less frequently traded token pairs and lead to large
bid-ask spreads and low trading volumes.

In our approach, we want to collect orders for a set of multiple tokens in a single joint order
book and compute exchange prices for all token pairs simultaneously at the end of discrete time
intervals (\emph{multi-token batch auction}).
Trades between the same token pairs are then all executed at the same exchange rate (\emph{uniform
clearing price}).
Moreover, this mechanism enables so-called \emph{ring trades}, where orders are matched along
cycles of tokens.
In order to exclude arbitrage opportunities, we require prices to be consistent along such cycles,
\ie we want the constraint
\begin{align}
  p_{A|B} \cdot p_{B|C} = p_{A|C}
  \label{eq:arbitrage_freeness}
\end{align}
to be satisfied for the prices of all tokens A,B,C.

In this document, we want to describe the problem of determining uniform clearing prices for all
pairs of tokens involved in a multi-token batch auction process, as well as present a mixed-integer
programming solution approach.


\clearpage
\section{Problem statement}
\label{sec:problem}

\subsection{Data}
\label{subsec:data}

Let $ \tokens := \{ \tau_0 \ldots \tau_{n-1} \} $ denote the set of the $ n $ tokens that we want
to consider.
The token~$ \tau_0 $ shall be referred to as \emph{reference token}, which will become important in
our modelling approaches later on in \secref{sec:models}.
For convenience of notation, we use $ \mathcal{I}^t := \{ 0 \ldots n-1 \} $ to denote the indices
of our token set.

Pairwise exchange rates between two tokens $ \tau_i $ and $ \tau_j $ will be denoted by
$ p_{i|j} $, meaning the price of one unit of $ \tau_i $ measured in units of $ \tau_j $.
As an example, if $ p_{i|j} = 10 $, one would need to pay an amount of $ 10 $ units of $ \tau_j $
in order to purchase one unit of $ \tau_i $.

Let there be a set $ \orders = \{ \omega_1 \ldots \omega_N \} $ of $ N $ limit orders in the batch
to be processed and let $ \ibuyorders, \isellorders \subseteq \{ 1 \ldots N \} =: \iorders $ denote
the sets of indices of buy and sell orders, respectively.
Every order must be of exactly one of the two types, so $ \ibuyorders \cap \isellorders =
\emptyset $ and $ \ibuyorders \cup \isellorders = \{ 1 \ldots N \} $.

Moreover, every buy order consists of a tuple $ (\tau_{j_b},\tau_{j_s},\overline{x},\pi) $ that is
to be read as
\vspace{-.6cm}
\begin{center}
  \emph{
    "Buy (at most) $ \overline{x} $ units of token $ \tau_{j_b} $ for token $ \tau_{j_s} $
    if the rate $ p_{j_b|j_s} $ is at most $ \pi $".
  }
\end{center}
\vspace{-.4cm}
Analogously, every sell order is represented by a tuple
$ (\tau_{j_s},\tau_{j_b},\overline{y},\pi) $ with the meaning
\vspace{-.6cm}
\begin{center}
  \emph{
    "Sell (at most) $ \overline{y} $ units of token $ \tau_{j_s} $ for token $ \tau_{j_b} $
    if the rate $ p_{j_s|j_b} $ is at least $ \pi $".
  }
\end{center}
\vspace{-.3cm}

The semantics of the two order types are somewhat similar in that there is one token to be
exchanged against another, if a certain limit price is guaranteed.
Such orders are hence referred to as \emph{limit orders}.
The difference between buy and sell orders is the following:
\begin{itemize}
  \item In a buy order, a fixed (maximum) buy volume $ \overline{x} $ is set, so the buyer
  indicates a preference to buy $ \overline{x} $ units of token $ \tau_{j_b} $.
  The amount of units of token $ \tau_{j_s} $ to be sold is flexible, as long as the limit price is
  respected (the smaller, the better).
  \item In a sell order, a fixed (maximum) sell volume $ \overline{y} $ is set, so the seller
  indicates a preference to sell $ \overline{y} $ units of token $ \tau_{j_s} $.
  The amount of units of token $ \tau_{j_b} $ to be bought is flexible, as long as the limit price
  is respected (the higher, the better).
\end{itemize}
\vspace{.3cm}

Nevertheless, the question remains whether it makes sense to keep the distinction between buy and
sell orders.
Consider a buy order $ (\tau_{j_b},\tau_{j_s},\overline{x},\pi) $, which reveals that the buyer
would be ready to sell at most $ \pi \cdot \overline{x} $ tokens $ \tau_{j_s} $ in exchange for
tokens $ \tau_{j_b} $.
Then, from an economically rational point of view, the buyer should not object to receiving more
than $ \overline{x} $ tokens $ \tau_{j_b} $ for a maximum of $ \pi \cdot \overline{x} $ tokens
$ \tau_{j_s} $, \ie if the exchange rate $ p_{j_b|j_s} $ is lower than $ \pi $.
Hence, the buy order could be translated into an equivalent sell order
$ (\tau_{j_s},\tau_{j_b},\pi \cdot \overline{x},\pi) $.
In this paper, however, we will continue to treat the different order types separately.



\subsection{Objectives}
\label{subsec:objectives}

Having collected a batch of buy and sell orders for our set of tokens, we ultimately want to
compute exchange rates for all token pairs.
Therefore, the following optimization criteria could be used:
\begin{itemize}
  \item maximize the amount of trade that is enabled (\wrt some reference token)
  \item maximize \emph{trader welfare}, \eg defined as difference of paid price vs. limit price
\end{itemize}


\subsection{Constraints}
\label{subsec:constraints}

The solution that we are aiming at needs to satisfy several requirements that can be stated on a
high level as follows:
\begin{itemize}
  \item \emph{(buy limit price)}:\\
  for all buy orders $ \omega_i \in \orders, \> i \in \ibuyorders $:
  the order can only be executed (fully or fractionally) if the exchange rate does not exceed the
  stated limit price, \ie $ p_{b|s} \le \pi $.
  \item \emph{(sell limit price)}:\\
  for all sell orders $ \omega_i \in \orders, \> i \in \isellorders $:
  the order can only be executed (fully or fractionally) if the exchange rate is not lower than the
  stated limit price, \ie $ p_{s|b} \ge \pi $.
  \item \emph{(token balance)}:\\
  for every token $ \tau_j \in \tokens $: the amount of tokens $ \tau_j $ that were bought must
  equal the amount of tokens $ \tau_j $ that were sold.
  \item \emph{(price coherence)}:\\
  for all token pairs $ (\tau_j,\tau_k) $: $ p_{j|k} \cdot p_{k|j} = 1 $.
  \item \emph{(arbitrage freeness)}:\\
  for all token triples $ (\tau_j,\tau_k,\tau_l) $: $ p_{j|k} \cdot p_{k|l} = p_{j|l} $.
\end{itemize}

\vspace{.6cm}
\begin{lemma}
  (arbitrage freeness) $ \Rightarrow $ (price coherence).
\end{lemma}
\vspace{-.8cm}
\begin{proof}
  Consider three tokens $ \{\tau_j,\tau_k,\tau_l\} \subset \tokens $.
  We apply the arbitrage-freeness condition twice:
  \begin{itemize}
    \item[(i)] $ p_{j|k} \cdot p_{k|l} = p_{j|l} $
    \item[(ii)] $ p_{j|l} \cdot p_{l|k} = p_{j|k} $
  \end{itemize}
  Inserting (i) into (ii) yields
  \begin{align*}
    p_{j|k} \cdot p_{k|l} \cdot p_{l|k} = p_{j|k}
    \qquad \Leftrightarrow \qquad
    p_{k|l} \cdot p_{l|k} = 1
  \end{align*}
\end{proof}
\vspace{-.4cm}

Notice that the above constraints also imply $ p_{j|j} = 1 $ for every
token~$ \tau_j \in \tokens $.

\subsection{Properties}
\label{subsec:properties}

(Before presenting different modelling approaches and formulations for the batch auction problem,
we first want to analyse/show some important properties.)

Here is a list of research questions that we would like to answer.

\begin{itemize}
  \item In an optimal solution, is it guaranteed that orders can always be fully executed if their
  limit prices are strictly higher (buy order) or lower (sell order) than the exchange rate between
  the respective token pair?
  If true, we would only need to resort to partially executing orders if the execution price is
  equal to the limit price.
  \item Does the optimal solution depend on the choice of the reference token, or not?
  \item What is the impact of optimizing the trading volume vs. the traders' welfare?
\end{itemize}

\vspace{1cm}
%\begin{lemma}
%  (arbitrage freeness) $ \Rightarrow $ (price coherence).
%\end{lemma}
%\vspace{-.5cm}
%\begin{proof}
%  Consider three tokens $ \{\tau_j,\tau_k,\tau_l\} \subset \tokens $.
%  We apply the arbitrage-freeness condition twice:
%  \begin{itemize}
%    \item[(i)] $ p_{j|k} \cdot p_{k|l} = p_{j|l} $
%    \item[(ii)] $ p_{j|l} \cdot p_{l|k} = p_{j|k} $
%  \end{itemize}
%  Inserting (i) into (ii) yields
%  \begin{align*}
%    p_{j|k} \cdot p_{k|l} \cdot p_{l|k} = p_{j|k}
%    \qquad \Leftrightarrow \qquad
%    p_{k|l} \cdot p_{l|k} = 1
%  \end{align*}
%\end{proof}


\clearpage
\section{Models and Formulations}
\label{sec:models}

In this section, we want to present and discuss several solution approaches for our batch auction
problem, mainly in terms of mathematical optimization formulations.
We will begin with a nonlinear programming formulation (NLP) and proceed with mixed-integer linear
(MIP) and network flow formulations thereafter.

With the exchange rates between pairs of tokens being variables, modelling arbitrage-freeness
constraints of type \eqref{eq:arbitrage_freeness} intuitively leads to many multiplications being
required.
This may result in unnecessary limitations to the problem size that is tractable as well as
numerical instability.
However, we can do better by not considering all pairwise token rates explicitely but representing
all token prices only with respect to a single reference token $ \tau_0 $.
Let $ p_j := p_{j|0} $ denote the price of token $ \tau_j $ expressed in units of token $ \tau_0 $ 
(hence, $ p_0 = 1 $).
Applying the arbitrage-freeness and price-coherence conditions directly, we can express the
exchange rate between two tokens $ \tau_j $ and $ \tau_k $ as
\begin{align}
  p_{j|k} = p_{j|0} \cdot p_{0|k} = \frac{p_{j|0}}{p_{k|0}} = \frac{p_j}{p_k}.
\end{align}

\subsubsection*{Data}

In the following, we will express all information given in the orders in terms of data
matrices and vectors, aiming at being able to write down a complete optimization model.

We introduce two data matrices
\begin{align*}
  \mathbf{T}^b &\in \{0,1\}^{N \times n} && \text{with} & \mathbf{T}^b \ni t^b_{i,j} = 1
  &\Leftrightarrow
  \text{token} \; \tau_j \; \text{to be bought in order} \; \omega_i \\
  \mathbf{T}^s &\in \{0,1\}^{N \times n} && \text{with} & \mathbf{T}^s \ni t^s_{i,j} = 1
  &\Leftrightarrow
  \text{token} \; \tau_j \; \text{to be sold in order} \; \omega_i
\end{align*}
As for now, we are only considering orders of one token type against one other, so there must be
exactly one entry equal to $ 1 $ per row (order) in both $ \mathbf{T}^b $ and $ \mathbf{T}^s $.

The maximum number of units of tokens to be bought in an order $ \omega_i $ shall be denoted by
$ \overline{x}_i $ and stored, for all orders, in a vector
$ \overline{\mathbf{x}} \in \mathbb{R}^N_{\ge 0} \cup \{+\infty\} $.
Generically, $ \overline{x}_i $ takes a finite value if $ \omega_i $ is a limit buy order
(\ie $ i \in \ibuyorders $).
In case $ \omega_i $ is a sell order (\ie $ i \in \isellorders $), $ \overline{x}_i $ can be set to
infinity, since the seller does specify an upper limit on the tokens to be received.

Similarly, let
$ (\overline{y}_i) =: \overline{\mathbf{y}} \in \mathbb{R}^N_{\ge 0} \cup \{+\infty\} $ contain the
maximum amounts of tokens to be sold in every order.
For all limit sell orders $ \omega_i $, we have $ \overline{y}_i < \infty $.
On the other hand, if $ \omega_i $ is a limit buy order, we can either set
$ \overline{y}_i = +\infty $ and rely on the implicit bound given via the limit price $ \pi_i $ (as
defined below), or state it explicitely as $ \overline{y}_i = \overline{x}_i \cdot \pi_i $.

The limit prices of all orders shall be stored as vector $ (\pi_i) =: \bm{\pi} \in \mathbb{R}^N_
{\ge 0} $, where $ \pi_i \in \bm{\pi} $ refers to the exchange rate between the respective buy and
sell tokens at which order $ \omega_i $ may be executed (according to the definition in
\subsecref{subsec:data}).


\subsection{Nonlinear programming model}
\label{subsec:NLPmodel}

The first model that we developed involves nonlinear constraints through multiplication of token
amounts and prices, but does not require binary variables.

\subsubsection*{Variables}

We represent all token prices in a vector $ \mathbf{p} \in \mathbb{R}^n_{\ge 0} $, \ie $ p_j $
denotes the price of token $ \tau_j $ \wrt the reference token $ \tau_0 $.
It makes sense to incorporate some explicit lower and upper bound for the price of every token $
\tau_j $ (for instance, in order to avoid excessive fluctuations), so let us require
$ p_j \in [\underline{p}_j,\overline{p}_j] $.
In our case, we use a maximum fluctuation parameter $ \delta \ge 0 $ to bound the deviation of the
computed exchange rates from the previous ones.
In particular, let $ p^\mathrm{old}_j $ be the price of token $ \tau_j $ found in the previous
batch auction iteration.
We then require $ p_j \in [(\frac{1}{1+\delta}) p^\mathrm{old}_j, (1+\delta) \> p^\mathrm{old}_j] $
for all tokens $ \tau_j \neq \tau_0 $.
However, this only limits fluctuations on token pairs involving the reference token to the desired
extent.
In order to guarantee the same maximum fluctuation on all other token pairs $ (\tau_j,\tau_k) $,
we further add the condition
\begin{align*}
  p_{j|k} \in
  \left[
    \left(\frac{1}{1+\delta}\right) p^\mathrm{old}_{j|k},
    (1+\delta) \, p^\mathrm{old}_{j|k}
  \right]
  \quad &\Leftrightarrow \quad
  \frac{p_j}{p_k} \in
  \left[
    \left(\frac{1}{1+\delta}\right) \frac{p^\mathrm{old}_j}{p^\mathrm{old}_k},
    (1+\delta) \, \frac{p^\mathrm{old}_j}{p^\mathrm{old}_k}
  \right]\\[2mm]
  \quad &\Leftrightarrow \quad
  \begin{cases}
    p_j \ge \left(\frac{1}{1+\delta}\right) \frac{p^\mathrm{old}_j}{p^\mathrm{old}_k} \, p_k
    \\[2mm]
    p_j \le \left(1+\delta\right) \frac{p^\mathrm{old}_j}{p^\mathrm{old}_k} \, p_k
  \end{cases}.
\end{align*}
As an example, $ \delta = 1 $ would set the bounds to half/double the previous exchange rates.
If no previous price exists for some token $ \tau_j $, the price bounds $ \underline{p}_j $
and $ \overline{p}_j $ could be determined on the basis of the limit prices given in the orders
that involve $ \tau_j $.
In any case, we set $ \underline{p}_0 = \overline{p}_0 = 1 $ in order to fix the price of
$ \tau_0 $ to $ p_0 = 1 $.
All price bounds shall be stored in vectors $ \underline{\mathbf{p}} $ and $ \overline{\mathbf{p}}
$, respectively.

For the executed volumes of all orders, we define a vector $ \mathbf{v} \in \mathbb{R}^N $, where
$ v_i \in \mathbf{v} $ contains the traded volume of order $ \omega_i $ in terms of units of the
reference token $ \tau_0 $.

The number of tokens bought in an order $ \omega_i $ shall be denoted by $ x_i $, with
$ (x_i) =: \mathbf{x} \in \mathbb{R}^N_{\ge 0} $.
Conversely, $ (y_i) =: \mathbf{y} \in \mathbb{R}^N_{\ge 0} $ represents the amount of tokens sold
per order.

\subsubsection*{Model}

\begin{subequations}
\begin{align}
  \text{maximize} \quad & \sum\limits_{i \in \iorders} v_i
  \label{eq:nlp_objective}
  \\[4mm]
  \text{subject to} \quad
  \sum\limits_{i \in \iorders} t^b_{i,j} \, x_i
  &= \sum\limits_{i \in \iorders} t^s_{i,j} \, y_i
  && \forall \> j \in \itokens
  \label{eq:nlp_tokenbalance}
  \\[4mm]
  v_i
  &= x_i \sum\limits_{j \in \itokens} t^b_{i,j} \, p_j
  && \forall \> i \in \iorders
  \label{eq:nlp_buyvolume}
  \\[2mm]
  v_i
  &= y_i \sum\limits_{j \in \itokens} t^s_{i,j} \, p_j
  && \forall \> i \in \iorders
  \label{eq:nlp_sellvolume}
  \\[4mm]
  x_i &\le \overline{x}_i
  && \forall \> i \in \ibuyorders
  \label{eq:nlp_tokenamount_1}
  \\[1mm]
  y_i &\le x_i \, \pi_i
  && \forall \> i \in \ibuyorders
  \label{eq:nlp_tokenamount_2}
  \\[4mm]
  y_i &\le \overline{y}_i
  && \forall \> i \in \isellorders
  \label{eq:nlp_tokenamount_3}
  \\[1mm]
  x_i &\ge y_i \, \pi_i
  && \forall \> i \in \isellorders
  \label{eq:nlp_tokenamount_4}
  \\[4mm]
  x_i, y_i, v_i &\in \mathbb{R}_{\ge 0}
  && \forall \> i \in \iorders
\end{align}
\label{eq:nlp}
\end{subequations}

The objective function \eqref{eq:nlp_objective} maximizes the total volume in terms of units of
the reference token $ \tau_0 $ that is processed with all orders.

Constraint \eqref{eq:nlp_tokenbalance} ensures that the total numbers of tokens bought and sold are
equal for every token across all orders.
The summations in this constraint are only responsible for selecting the correct tokens that are
traded in the orders.

The constraints \eqref{eq:nlp_buyvolume} and \eqref{eq:nlp_sellvolume} compute the buy and sell
trade volume for every order \wrt the reference token, and make sure these two are equal.
This guarantees that the token prices are chosen such that they are consistent with the traded
amounts of tokens.
If the traded token amounts $ x_i $ and $ y_i $ are zero for some order $ \omega_i $, \ie
$ \omega_i $ is not executed at all, the corresponding trade volume $ v_i $ will be zero as well.
However, this comes at the price of introducing nonlinearity (and even nonconvexity) into the
model.

Finally, the limits in terms of token amounts to be bought/sold in a limit order are incorporated
into the model via the constraints \eqref{eq:nlp_tokenamount_1}--\eqref{eq:nlp_tokenamount_4}.

One major weakness of the model is the fact that orders can be left unexecuted even if the computed
token prices satisfy the given limit price.
We believe that this can only be cured with the introduction of binary variables that indicate
whether prices allow for an order to be executed, or not.


\subsection{Mixed-integer linear programming model I}
\label{subsec:MIPmodel_1}

As an alternative to the NLP model presented above, we are now going to propose a
mixed-integer linear programming formulation.

\subsubsection*{Variables}

In addition to the same price and volume variables as in the NLP model \eqref{eq:nlp},
$ \mathbf{p} \in \mathbb{R}^n $ and $ \mathbf{v} \in \mathbb{R}^N $, respectively, the MIP model
requires several other variables.
First and foremost, let $ \mathbf{z} \in \{0,1\}^N $ be a vector of binary variables, where $ z_i
\in \mathbf{z} $ indicates whether order $ \omega_i $ may be (fully or partially) executed, or not.
Precisely, we require $ z = 1 $ if and only if the prices of the tokens that are present in the
order satisfy the respective limit price $ \pi_i $, otherwise $ z = 0 $.

The feasible region for the execution volume $ v_i $ of an order $ \omega_i $ depends on the value
of $ z_i $.
In particular, $ v_i $ must be set to zero if $ z_i = 0 $ and can only be non-zero otherwise.
In order to model this disjoint behaviour, we make use of a \emph{disjunctive programming}
formulation that needs the following auxiliary non-negative price variable vectors:
\begin{align*}
  \mathbf{p}^{b,0}, \mathbf{p}^{b,1}, \mathbf{p}^{s,0}, \mathbf{p}^{s,1} \in \mathbb{R}^N_{\ge 0}.
\end{align*}

\subsubsection*{Parameters}

The model allows for setting a minimum and maximum fraction of execution for every order.
For now, we will use global values $ 0 \le \underline{r} \le \overline{r} \le 1 $ and for all
current purposes set $ \overline{r} = 1 $.


\subsubsection*{Model}

\begin{subequations}
\begin{align}
  \text{maximize} \quad & \sum\limits_{i \in \iorders} v_i
  \label{eq:mip_objective}
  \\[4mm]
  \text{subject to} \quad
  \sum\limits_{i \in \iorders} t^b_{i,j} \, v_i
  &= \sum\limits_{i \in \iorders} t^s_{i,j} \, v_i
  && \forall \> j \in \itokens
  \label{eq:mip_tokenbalance}
  \\[4mm]
  \sum\limits_{j \in \itokens} t^b_{i,j} \, \underline{p}_j (1-z_i)
  &\le p_i^{b,0} \le  \sum\limits_{j \in \itokens} t^b_{i,j} \, \overline{p}_j (1-z_i)
  && \forall \> i \in \iorders
  \label{eq:mip_bounds_buyprice_0}
  \\[1mm]
  \sum\limits_{j \in \itokens} t^s_{i,j} \, \underline{p}_j (1-z_i)
  &\le p_i^{s,0} \le  \sum\limits_{j \in \itokens} t^s_{i,j} \, \overline{p}_j (1-z_i)
  && \forall \> i \in \iorders
  \label{eq:mip_bounds_sellprice_0}
  \\[1mm]
  \sum\limits_{j \in \itokens} t^b_{i,j} \, \underline{p}_j z_i
  &\le p_i^{b,1} \le  \sum\limits_{j \in \itokens} t^b_{i,j} \, \overline{p}_j z_i
  && \forall \> i \in \iorders
  \label{eq:mip_bounds_buyprice_1}
  \\[1mm]
  \sum\limits_{j \in \itokens} t^s_{i,j} \, \underline{p}_j z_i
  &\le p_i^{s,1} \le  \sum\limits_{j \in \itokens} t^s_{i,j} \, \overline{p}_j z_i
  && \forall \> i \in \iorders
  \label{eq:mip_bounds_sellprice_1}
  \\[4mm]
  \sum\limits_{j \in \itokens} t^b_{i,j} \, p_j
  &= p_i^{b,0} + p_i^{b,1}
  && \forall \> i \in \iorders
  \label{eq:mip_aggr_buyprice}
  \\[1mm]
  \sum\limits_{j \in \itokens} t^s_{i,j} \, p_j
  &= p_i^{s,0} + p_i^{s,1}
  && \forall \> i \in \iorders
  \label{eq:mip_aggr_sellprice}
  \\[4mm]
  p^{b,0}_i
  &\le \pi_i \, p^{s,0}_i (1-\varepsilon)
  && \forall \> i \in \ibuyorders
  \label{eq:mip_buyorder_disj_0}
  \\[1mm]
  p^{b,1}_i
  &\ge \pi_i \, p^{s,1}_i
  && \forall \> i \in \ibuyorders
  \label{eq:mip_buyorder_disj_1}
  \\[1mm]
  v_i
  &\ge \underline{r} \, \overline{x}_i \, p^{b,1}_i
  && \forall \> i \in \ibuyorders
  \label{eq:mip_buyorder_volume_lb}
  \\[1mm]
  v_i
  &\le \overline{r} \, \overline{x}_i \, p^{b,1}_i
  && \forall \> i \in \ibuyorders
  \label{eq:mip_buyorder_volume_ub}
  \\[4mm]
  p^{s,0}_i
  &\le \pi_i \, p^{b,0}_i (1-\varepsilon)
  && \forall \> i \in \isellorders
  \label{eq:mip_sellorder_disj_0}
  \\[1mm]
  p^{s,1}_i
  &\ge \pi_i \, p^{b,1}_i
  && \forall \> i \in \isellorders
  \label{eq:mip_sellorder_disj_1}
  \\[1mm]
  v_i
  &\ge \underline{r} \, \overline{y}_i \, p^{s,1}_i
  && \forall \> i \in \isellorders
  \label{eq:mip_sellorder_volume_lb}
  \\[1mm]
  v_i
  &\le \overline{r} \, \overline{y}_i \, p^{s,1}_i
  && \forall \> i \in \isellorders
  \label{eq:mip_sellorder_volume_ub}
  \\[4mm]
  z_i
  &\in \{0,1\}
  && \forall \> i \in \iorders
  \\[1mm]
  v_i, \, p^{b,0}_i, \, p^{b,1}_i, \, p^{s,0}_i, \, p^{s,1}_i
  &\in \mathbb{R}_{\ge 0}
  && \forall \> i \in \iorders
  \\[1mm]
  p_j
  &\in \mathbb{R}_{\ge 0}
  && \forall \> j \in \itokens
\end{align}
\label{eq:mip1}
\end{subequations}

The objective function \eqref{eq:mip_objective} maximizes the total volume in terms of units of
the reference token $ \tau_0 $ that is processed with all orders.

Constraint \eqref{eq:mip_tokenbalance} secures that the total buy and sell volumes across all 
orders must be equal for every token.
With uniform clearing prices, volume balance implies token balance.

The auxiliary variables in $ \mathbf{p}^{b,0} $ and $ \mathbf{p}^{s,0} $ refer to the prices of the
buy- and sell-token of every order $ \omega_i $ if that order is not executed ($ z_i = 0 $).
In that case, their values must then lie within in the respective bounds provided by $ \underline{
\mathbf{p}} $ and $ \overline{\mathbf{p}} $, and otherwise shall be set to zero.
This requirement is ensured by the constraints \eqref{eq:mip_bounds_buyprice_0} and
\eqref{eq:mip_bounds_sellprice_0}.
Note that the summations in the constraints are only needed to ensure that the right variable is
selected from the respective variable vector.

Similarly as above, the constraints \eqref{eq:mip_bounds_buyprice_1} and
\eqref{eq:mip_bounds_sellprice_1} control the auxiliary variables in $ \mathbf{p}^{b,1} $ and $ 
\mathbf{p}^{s,1} $ for the case that an order $ \omega_i $ is executed ($ z_i = 1 $).

The relation between the auxiliary price variables and the actual token prices is established by
the constraints \eqref{eq:mip_aggr_buyprice} and \eqref{eq:mip_aggr_sellprice}.

For buy orders, the disjunctive behaviour is modelled by the constraints
\eqref{eq:mip_buyorder_disj_0}--\eqref{eq:mip_buyorder_volume_ub}.
The idea is as follows:
If some order $ \omega_i $ is not to be executed ($ z_i = 0 $), then the prices
$ p_i^{b,0} $ and $ p_i^{s,0} $ must both lie within the bounds given by
\eqref{eq:mip_bounds_buyprice_0} and \eqref{eq:mip_bounds_sellprice_0} as well as fulfill
\eqref{eq:mip_buyorder_disj_0} (\ie not satisfy the limit price).
We want to ensure that the token prices are at least a little margin off the stated limit price in
that case, hence the multiplication with $ (1-\varepsilon) $.
At the same time, $ p_i^{b,1} $ and $ p_i^{s,1} $ are set to zero by
\eqref{eq:mip_bounds_buyprice_1} and \eqref{eq:mip_bounds_sellprice_1}, thus trivially satisfying
\eqref{eq:mip_buyorder_disj_1}.
This then also implies the volume $ v_i $ to be set to zero by \eqref{eq:mip_buyorder_volume_lb}
and \eqref{eq:mip_buyorder_volume_ub}.
Conversely, if $ \omega_i $ is to be executed ($ z_i = 1 $), $ p_i^{b,0} $ and $ p_i^{s,0} $ are
set to zero by \eqref{eq:mip_bounds_buyprice_0} and \eqref{eq:mip_bounds_sellprice_0}, while
$ p_i^{b,1} $ and $ p_i^{s,1} $ lie within the bounds provided by \eqref{eq:mip_bounds_buyprice_1}
and \eqref{eq:mip_bounds_sellprice_1} as well as fulfill \eqref{eq:mip_buyorder_disj_1} (\ie
satisfy the limit price).
This finally requires the execution volume $ v_i $ to be within the specified fractions via the
constraints \eqref{eq:mip_buyorder_volume_ub} and \eqref{eq:mip_buyorder_volume_ub}.

The reasoning is analogous for sell orders and expressed by
\eqref{eq:mip_sellorder_disj_0}--\eqref{eq:mip_sellorder_volume_ub}.


\subsubsection*{Computational results}
%\label{subsec:computational_results}

The following \tabref{tab:mip_results} should give an indication of the solving times of the model
\wrt different numbers of tokens and orders.
Every entry in the table represents the mean solving time of a set of 10 instances.
The buy/sell token quantities and limit prices of the orders have been randomly generated.
In order to guarantee feasibility for all instances, we have selected a value of $ \underline{r} =
0 $, so the solver may choose to execute 0\% of an order even if the token prices comply with the
limit price.
We have modelled the problem using \textsc{Python}/\textsc{Pyomo} and employed
\textsc{Gurobi}~7.5.2 as a solver.

\begin{table}
  \centering
  \begin{tabular}{ccrrrr}
    \toprule
    && \multicolumn{4}{c}{\# orders per token pair}\\
    \cmidrule{3-6}
    \# tokens & (\# pairs) &   4  &   8  &    20 &     40 \\
    \midrule
    2         &       (1)  & 0.05 & 0.06 &  0.08 &   0.10 \\
    3         &       (3)  & 0.07 & 0.12 &  0.31 &   0.89 \\
    4         &       (6)  & 0.13 & 0.24 &  0.74 &   3.09 \\
    5         &       (10) & 0.25 & 0.69 &  4.22 &  84.36 \\
    6         &       (15) & 0.40 & 2.04 & 77.32 & 684.40 \\
    \bottomrule
  \end{tabular}
  \caption{Solving times in [s] of the MIP formulation to global optimality.}
  \label{tab:mip_results}
\end{table}

It can be seen that the running times increase sharply for 5 or more tokens and the many-order
cases.
One influencing factor might be the equal number of orders for all token pairs in our testset,
which leads all token prices being maximally interconnected and which might not occur as such in
practice.
On the other hand, the more orders there are on some token pair, the denser the stated limit prices
become, which might make it possible to aggregate orders as a means of preprocessing.


\subsection{Mixed-integer linear programming model II}
\label{subsec:MIPmodel_2}

The previous MIP model \eqref{eq:mip1} considers all orders independent from each other, \ie the
decision whether some order should be executed or not is not explicitely connected to the decision
for other orders.
The connection is only indirectly established through the propagation of feasible price ranges.
In practice, however, there is more structure to the execution of orders, particularly within the
same token pair.
For instance, if some sell order with limit price $ \pi^* $ is enabled by the current choice of
prices, all other sell orders with higher limit prices should be executed as well (and vice-versa
for buy orders).
This insight gives rise to an alternative MIP formulation that we will present in the following.


\subsubsection*{Data}

We want to aggregate orders on every pair of tokens that is traded, so let the set of token index
pairs be denoted by $ \itokenpairs := \{(j,k) \in \itokens \times \itokens, j \neq k\} $, with
$ (j,k) \in \itokenpairs $ representing the pair of tokens $ \tau_j $ and $ \tau_k $.
Notice that we consider the token pairs to be ordered tuples, so $ (j,k) \in \itokenpairs $ and
$ (k,j) \in \itokenpairs $ are treated separately.

For every token pair $ (j,k) \in \itokenpairs $, we collect all buy and sell orders
$ (\tau_j,\tau_k,\, \cdot \,,\, \cdot \,) $ that were submitted for $ (j,k) $, and sort them in an
increasing order by their limit prices.
Assuming that there are $ m $ orders for token pair $ (j,k) $, we get
\begin{align}
  \underline{p}_{j|k} =:
    \pi^{(0)}_{j,k} < \pi^{(1)}_{j,k} < \pi^{(2)}_{j,k} <
    \ldots < \pi^{(m)}_{j,k} < \pi^{(m+1)}_{j,k}
  := \overline{p}_{j|k},
  \label{eq:limit_price_ordering}
\end{align}
where $ \underline{p}_{j|k} $ and $ \overline{p}_{j|k} $ are explicit but somewhat arbitrary bounds
on the exchange rate (\eg half and double the previous rate).
Notice, moreover, that orders of the same type (buy/sell) with the same limit price are aggregated
into one single order beforehand, such that the strict inequalities always hold.
The above ordering \eqref{eq:limit_price_ordering} defines regions
$ r_l := [\pi^{(l)},\pi^{(l+1)}] $, $ l \in \{0 \ldots m\} $, for the exchange rate $ p_{j|k} $.
Let $ \mathcal{I}_{j,k}^r := \{0 \ldots m\} $ denote the set of such regions for every token pair
$ (j,k) $, \ie $ l \in \mathcal{I}_{j,k}^r $ refers to the interval $ r_l $ of that token pair.

Now, let $ \tilde{x}_{j,k,l}^b $ denote to the cumulated amount of tokens $ \tau_j $ that could be
bought across all buy orders of token pair $ (j,k) $, if the exchange rate $ p_{j|k} $ were in
$ r_l $.
Similarly, $ \tilde{y}_{j,k,l}^s $ shall refer to the amount of tokens $ \tau_j $ to be sold across
all sell orders under the same condition.
Let $ \tilde{\mathbf{x}} := (\tilde{x}_{j,k,l}) $ and
$ \tilde{\mathbf{y}} := (\tilde{y}_{j,k,l}) $.
From here on, we are only working with this aggregated order representation.


\subsubsection*{Variables}

We will use the same price variables $ \mathbf{p} \in \mathbb{R}^n_{\ge 0} $ as previously.
In addition, we will use variables $ (v_{j,k}^\mathrm{abs}) =: \mathbf{v}^\mathrm{abs} \in 
\mathbb{R}_{\ge 0}^{\mid \itokenpairs \mid} $ and
$ (v_{j,k}^\mathrm{net}) =: \mathbf{v}^\mathrm{net} \in \mathbb{R}_{\ge 0}^{\mid \itokenpairs \mid}
$ to represent the total absolute and net volume (in terms of units of token $ \tau_0 $) that is
traded on each token pair.
Furthermore, for every token pair $ (j,k) $ and every exchange rate interval $ r_l $,
$ l \in \mathcal{I}_{j,k}^r $, we introduce a binary variable $ z_{j,k,l} \in \{0,1\} $ that
indicates whether the exchange rate $ p_{j|k} = \frac{p_j}{p_k} $ lies within the interval $ r_l $
or not.
Let $ \mathbf{z} := (z_{j,k,l}) $.
In order for the disjunctive MIP formulation to work, we again need a set of auxiliary 
(disaggregated) variables.
Let $ \mathbf{v}^b := (v^b_{j,k,l}) $ and $ \mathbf{v}^s := (v^s_{j,k,l}) $ represent the volumes
that are traded in buy and sell orders, respectively, for every token pair and every exchange rate
interval.
Moreover, $ \mathbf{p}^{(1)} := (p^{(1)}_{j,k,l}) $ and $ \mathbf{p}^{(2)} := (p^{(2)}_{j,k,l}) $
shall denote the disaggregated price variables for the two respective tokens of every token pair
and all price intervals.

\subsubsection*{Parameters}

This model allows for the same parameters $ \underline{r} $ and $ \overline{r} $ as the previous
MIP model \eqref{eq:mip1}.


\subsubsection*{Model}

\begin{subequations}
\begin{align}
  \text{maximize} \quad & \sum\limits_{(j,k) \in \itokenpairs} v_{j,k}^\mathrm{abs}
  \label{eq:mip2_objective}
  \\[2mm]
  \text{subject to} \quad
  \sum_{\substack{k \in \itokens \\ k \neq j}} v_{k,j}^\mathrm{net}
  &= \sum_{\substack{k \in \itokens \\ k \neq j}} v_{j,k}^\mathrm{net}
  && \forall \> j \in \itokens
  \label{eq:mip2_tokenbalance}
  \\[2mm]
  \sum\limits_{l \in \mathcal{I}_{j,k}^r} z_{j,k,l} &= 1
  && \forall \> (j,k) \in \itokenpairs
  \label{eq:mip2_price_range}
  \\[2mm]
  v_{j,k}^\mathrm{abs}
  &= \sum\limits_{l {}\in \mathcal{I}_{j,k}^r} \left( v^b_{j,k,l} + v^s_{j,k,l} \right)
  && \forall \> (j,k) \in \itokenpairs
  \label{eq:mip2_volume_aggr_abs}
  \\[1mm]
  v_{j,k}^\mathrm{net}
  &= \sum\limits_{l \in \mathcal{I}_{j,k}^r} \left( v^b_{j,k,l} - v^s_{j,k,l} \right)
  && \forall \> (j,k) \in \itokenpairs
  \label{eq:mip2_volume_aggr_net}
  \\[1mm]
  p_j
  &= \sum\limits_{l \in \mathcal{I}_{j,k}^r} p^{(1)}_{j,k,l}
  && \forall \> (j,k) \in \itokenpairs
  \label{eq:mip2_price1_aggr}
  \\[1mm]
  p_k
  &= \sum\limits_{l \in \mathcal{I}_{j,k}^r} p^{(2)}_{j,k,l}
  && \forall \> (j,k) \in \itokenpairs
  \label{eq:mip2_price2_aggr}
  \\[4mm]
  p^{(1)}_{j,k,l}
  &\ge p^{(2)}_{j,k,l} \, \pi_{j,k}^{(l)}
  && \forall \> (j,k) \in \itokenpairs, l \in \mathcal{I}_{j,k}^r
  \label{eq:mip2_price1_lb}
  \\[1mm]
  p^{(1)}_{j,k,l}
  &\le p^{(2)}_{j,k,l} \, \pi_{j,k}^{(l+1)}
  && \forall \> (j,k) \in \itokenpairs, l \in \mathcal{I}_{j,k}^r
  \label{eq:mip2_price1_ub}
  \\[1mm]
  p^{(2)}_{j,k,l}
  &\ge \underline{p}_k \, z_{j,k,l}
  && \forall \> (j,k) \in \itokenpairs, l \in \mathcal{I}_{j,k}^r
  \label{eq:mip2_price2_lb}
  \\[1mm]
  p^{(2)}_{j,k,l}
  &\le \overline{p}_k \, z_{j,k,l}
  && \forall \> (j,k) \in \itokenpairs, l \in \mathcal{I}_{j,k}^r
  \label{eq:mip2_price2_ub}
  \\[2mm]
  v^b_{j,k,l}
  &\ge \underline{r} \, \tilde{x}^b_{j,k,l} \, p^{(1)}_{j,k,l}
  && \forall \> (j,k) \in \itokenpairs, l \in \mathcal{I}_{j,k}^r
  \label{eq:mip2_buyvolume_lb}
  \\[1mm]
  v^b_{j,k,l}
  &\le \overline{r} \, \tilde{x}^b_{j,k,l} \, p^{(1)}_{j,k,l}
  && \forall \> (j,k) \in \itokenpairs, l \in \mathcal{I}_{j,k}^r
  \label{eq:mip2_buyvolume_ub}
  \\[1mm]
  v^s_{j,k,l}
  &\ge \underline{r} \, \tilde{y}^s_{j,k,l} \, p^{(1)}_{j,k,l}
  && \forall \> (j,k) \in \itokenpairs, l \in \mathcal{I}_{j,k}^r
  \label{eq:mip2_sellvolume_lb}
  \\[1mm]
  v^s_{j,k,l}
  &\le \overline{r} \, \tilde{y}^s_{j,k,l} \, p^{(1)}_{j,k,l}
  && \forall \> (j,k) \in \itokenpairs, l \in \mathcal{I}_{j,k}^r
  \label{eq:mip2_sellvolume_ub}
  \\[2mm]
  p_j
  &\in \mathbb{R}_{\ge 0}
  && \forall \> j \in \itokens
  \\[1mm]
  v_{j,k}^\mathrm{abs}, v_{j,k}^\mathrm{net}
  &\in \mathbb{R}_{\ge 0}
  && \forall \> (j,k) \in \itokenpairs
  \\[1mm]
  z_{j,k,l}
  &\in \{0,1\}
  && \forall \> (j,k) \in \itokenpairs, l \in \mathcal{I}_{j,k}^r
  \\[1mm]
  p^{(1)}_{j,k,l}, p^{(2)}_{j,k,l}, v^b_{j,k,l}, v^s_{j,k,l}
  &\in \mathbb{R}_{\ge 0}
  && \forall \> (j,k) \in \itokenpairs, l \in \mathcal{I}_{j,k}^r
\end{align}
\label{eq:mip2}
\end{subequations}

The objective function \eqref{eq:mip2_objective} maximizes the sum of the trading volumes over all
trading pairs and is supposed to yield the same value as the objective \eqref{eq:mip_objective} of
the previous MIP model.

The first constraint \eqref{eq:mip2_tokenbalance} secures, again, the token balance for every token
by requiring the net traded volumes to be balanced over all token pairs in which the respective
token is involved.

All other constraints are induced by the disjunctive formulation that our model is based on.
For every token pair, the solver may choose exactly one price range, which is indicated by the
constraint \eqref{eq:mip2_price_range}.
The purpose of the following constraints \eqref{eq:mip2_volume_aggr_abs}--
\eqref{eq:mip2_price2_aggr} is the aggregation of the disaggregated variables, both for token
prices and trading volumes.
This builds the connection between the auxiliary variables and the actual price and volume
variables.
A main feature of disjunctive programming is that the auxiliary variables belonging to some part of
the disjunction are set to zero if that part is not selected to be active, and that they take
some meaningful value otherwise.
This is expressed by the constraints \eqref{eq:mip2_price1_lb}--\eqref{eq:mip2_sellvolume_ub}.
Thereby, the main dependency between the binary and the auxiliary variables is given in 
\eqref{eq:mip2_price2_lb} and \eqref{eq:mip2_price2_ub}, where auxiliary prices on some disjunction
part are either set to zero or are restricted to be within (non-negative) bounds.
If they are set to zero, it is implied that all auxiliary variables for the same part are also set
to zero by the other constraints.
In the other case, the respective auxiliary variables are restricted to fulfill the semantics of
the model, which are similar to the previous MIP model \eqref{eq:mip1}.


\subsection{Mixed-integer linear programming model III}
\label{subsec:MIPmodel_3}

In \ref{subsec:MIPmodel_2}, we consider aggregated orders on directed token pairs, so the pair $ 
(\tau_i,\tau_j) $ exists as well as $ (\tau_j,\tau_i) $.
However, we can also aggregate further and only consider undirected token pairs
$ \{\tau_i,\tau_j\} $.

TODO!


\subsection{Min-cost flow model}
\label{subsec:MCFmodel}

???


\newpage
\subsection{Computational comparison}
\label{subsec:computational_comparison}

In order to investigate the performance of our MIP models, we have conducted computational
experiments for different numbers of tokens ($ n \in \{5,10,20,50\} $) and orders ($ N \in 
\{100,200,500\} $).
For every combination of $ n $ and $ N $, we have generated 20 random instances, whereby the
randomness reflects our expectation of somewhat realistic situations.
In particular, we expect not all tokens to be equally important in terms of trading volume and,
hence, to have varying numbers of orders on different token pairs.
We used Gurobi 8.0.0 as MIP solver on an Intel(R) Core(TM) i7-8550U CPU @ 1.80GHz machine with 16Gb
RAM and using 4 threads, and a timelimit set to 1800 seconds for every instance.

The (geometric) means of the runtimes have been computed only \wrt the instances that could be
solved to optimality before the timelimit. Conversely, the average optimality gap does not take
solved instances into account.

$ \Rightarrow \underline{p}=0.5, \overline{p}=2.0, \underline{r}=0.2 $

\todo[inline]{TODO!}\vspace{-5mm}
The results of our computational experiments as in \tabref{tab:mip1_results},
\tabref{tab:mip2_results} and \tabref{tab:mip3_results} show that runtimes of the MIP formulation
sharply increase both with the number of assets and the number of orders that are being considered.
Several instances with more than 20 assets and more than 200 orders could not even be solved to
optimality within the timelimit.
If larger problems are to be considered, we can think of the following heuristics/approximations:
\begin{itemize}
    \item Aggregate orders with similar limit prices on every asset pair
    \item Optimize over subsets of assets separately and fix prices in overall problem
\end{itemize}

\vspace{.6cm}
\begin{table}[ht!]
  \centering
  \begin{tabular}{C{16mm}R{20mm}R{15mm}R{15mm}R{15mm}R{15mm}}
    \toprule
    && \multicolumn{4}{c}{\# assets}\\
    \cmidrule{3-6}
    \# orders &                         &    5  &   10  &     20  &     50  \\
    \midrule
    100       & $ \varnothing $ runtime &  0.32 &  0.32 &    0.90 &    5.54 \\
              & \# timeouts             &     0 &     0 &       0 &       0 \\
              & -- $ \varnothing $ gap  &     - &     - &       - &       - \\
    \midrule
    200       & $ \varnothing $ runtime &  1.31 &  2.49 &   26.25 &  571.10 \\
              & \# timeouts             &     0 &     0 &       0 &      12 \\
              & -- $ \varnothing $ gap  &     - &     - &       - & 23.46\% \\
    \midrule
    500       & $ \varnothing $ runtime & 15.67 & 82.47 &  522.72 &       . \\
              & \# timeouts             &     0 &     0 &      11 &       - \\
              & -- $ \varnothing $ gap  &     - &     - & 21.59\% &       - \\
    \bottomrule
  \end{tabular}
  \caption{Computational results for MIP model I \eqref{eq:mip1}.}
  \label{tab:mip1_results}
\end{table}
\vspace{.5cm}

\begin{table}[ht!]
  \centering
  \begin{tabular}{C{16mm}R{20mm}R{15mm}R{15mm}R{15mm}R{15mm}}
    \toprule
    && \multicolumn{4}{c}{\# assets}\\
    \cmidrule{3-6}
    \# orders &                         &    5  &   10  &     20  &     50  \\
    \midrule
    100       & $ \varnothing $ runtime &     . &     . &        . &      . \\
              & \# timeouts             &     . &     . &        . &      . \\
              & -- $ \varnothing $ gap  &     . &     . &        . &      . \\
    \midrule
    200       & $ \varnothing $ runtime &     . &     . &        . &      . \\
              & \# timeouts             &     . &     . &        . &      . \\
              & -- $ \varnothing $ gap  &     . &     . &        . &      . \\
    \midrule
    500       & $ \varnothing $ runtime &     . &     . &        . &      . \\
              & \# timeouts             &     . &     . &        . &      . \\
              & -- $ \varnothing $ gap  &     . &     . &        . &      . \\
    \bottomrule
  \end{tabular}
  \caption{Computational results for MIP model II \eqref{eq:mip2}.}
  \label{tab:mip2_results}
\end{table}
\vspace{.5cm}

\begin{table}[ht!]
  \centering
  \begin{tabular}{C{16mm}R{20mm}R{15mm}R{15mm}R{15mm}R{15mm}}
    \toprule
    && \multicolumn{4}{c}{\# assets}\\
    \cmidrule{3-6}
    \# orders &                         &    5  &   10  &     20  &     50  \\
    \midrule
    100       & $ \varnothing $ runtime &     . &     . &        . &      . \\
              & \# timeouts             &     . &     . &        . &      . \\
              & -- $ \varnothing $ gap  &     . &     . &        . &      . \\
    \midrule
    200       & $ \varnothing $ runtime &     . &     . &        . &      . \\
              & \# timeouts             &     . &     . &        . &      . \\
              & -- $ \varnothing $ gap  &     . &     . &        . &      . \\
    \midrule
    500       & $ \varnothing $ runtime &     . &     . &        . &      . \\
              & \# timeouts             &     . &     . &        . &      . \\
              & -- $ \varnothing $ gap  &     . &     . &        . &      . \\
    \bottomrule
  \end{tabular}
  \caption{Computational results for MIP model III \eqref{eq:mip3}.}
  \label{tab:mip3_results}
\end{table}


\clearpage
\section{Extensions}
\label{sec:extensions}

The problem can possibly be extended in various directions:
\begin{itemize}
  \item Basket orders: Buy/sell a set of tokens for a set of other tokens at some limit price.
  \item Automated market makers: Instead of signaling discrete demand at a specific price with one
  order, those would allow to express continues demand function over a price range.
  \item if it becomes a problem to find a valid solution \emph{at all}, the optimization problem can
  be broadened to allow violations of the current constraints but measure the violation and
  include this to the objective function.
\end{itemize}




\bibliographystyle{plain}
\bibliography{literature}

\end{document}